\section{Plan de Actividades}

El desarrollo de \textbf{Abstraktor} seguirá un enfoque basado en metodologías ágiles, con iteraciones semanales que permitan una evolución incremental del sistema. La planificación y el seguimiento del trabajo se organizarán a través de un \textbf{backlog de tareas y features}.

\subsection{Gestión del Alcance y Tiempos}
Se utilizará un backlog organizado de la siguiente manera:
\begin{itemize}
    \item Se priorizarán las tareas en función de su impacto y dependencia con otros componentes.
    \item Cada tarea será estimada en términos de dificultad con valores \textbf{1, 2 o 3}, en lugar de tiempos específicos.
    \item Las iteraciones semanales permitirán ajustar el backlog dinámicamente en función del progreso y los hallazgos en pruebas.
\end{itemize}

\subsection{Desarrollo de Componentes}
El desarrollo de \textbf{Abstraktor} se dividirá en los siguientes componentes principales:

\begin{itemize}
    \item \textbf{Instrumentador}: Procesamiento del código fuente y generación de ejecutables con estado interno expuesto. Este componente será construido en \textbf{Rust y C}, ya que requiere el uso de herramientas de bajo nivel para interactuar directamente con el código y exponer el estado interno de la aplicación.
    \item \textbf{Extensión de Mallory}: Fork de la herramienta Mallory para permitir \textit{fuzzing} eficiente y extracción de estado interno. Esta extensión se desarrollará en \textbf{C}, el mismo lenguaje utilizado en la implementación original de Mallory.
    \item \textbf{Generador de EPAs}: Construcción y exportación de abstracciones a partir de la ejecución del protocolo. Este componente se desarrollará en \textbf{Python}, debido a su amplia disponibilidad de bibliotecas y herramientas para la generación de imágenes y la manipulación de datos.
    \item \textbf{CLI}: Encargada de orquestar los demás componentes del sistema. Será desarrollada en \textbf{Rust}, aprovechando sus bibliotecas para la creación de interfaces de línea de comandos.
\end{itemize}

Cada iteración semanal incluirá avances en uno o más de estos componentes según las prioridades definidas en el backlog. Se desarrollarán todos los componentes en simultáneo, tratando de priorizar funcionalidades \textit{end to end}.

\subsection{Planificación de hitos y entregas intermedias}

El desarrollo de \textbf{Abstraktor} se organizará en torno a hitos principales y entregas intermedias, priorizando la implementación progresiva de funcionalidades \textit{end-to-end}. A continuación se describen los hitos junto con una estimación de esfuerzo en semanas y los entregables asociados:

\begin{enumerate}
    \item \textbf{Investigación de Raft y generación de posibles abstracciones} (2 semanas): Estudio detallado del protocolo Raft y diseño inicial de abstracciones que representen sus propiedades clave de manera verificable. \textbf{Entregable:} Presentación técnica.
    
    \item \textbf{Investigación de Mallory} (2 semanas): Análisis de la herramienta Mallory para comprender su funcionamiento interno y evaluar las modificaciones necesarias para integrarla al flujo de trabajo de \textbf{Abstraktor}. \textbf{Entregable:} Presentación técnica.
    
    \item \texttt{prepare-binary} (6 semanas): Implementación de los subcomandos \texttt{get-targets} e\\ \texttt{instrument-code}, permitiendo analizar el código fuente y generar un binario instrumentado. \textbf{Entregable:} Código funcional.
        
    \item \texttt{run-mallory} (6 semanas): Integración inicial de la extensión de Mallory para ejecutar el binario instrumentado y capturar eventos. \textbf{Entregable:} Código funcional.
        
    \item \texttt{generate-epa} (versión inicial) (4 semanas): Desarrollo de una versión simplificada del generador de EPAs, exportando abstracciones en formato de texto legible. \textbf{Entregable:} Código funcional.
        
    \item \texttt{generate-epa} (mejorado) (4 semanas): Extensión del generador de EPAs para soportar exportación en formatos enriquecidos (PDFs, informes) y mejoras en la representación de abstracciones. \textbf{Entregable:} Código funcional.
    
    \item \textbf{Validación y refinamiento de abstracciones} (4 semanas): Evaluación de la herramienta utilizando implementaciones conocidas de Raft, identificación de inconsistencias y ajustes en las abstracciones generadas. \textbf{Entregable:} Código funcional.

    \item \textbf{Generación de documentación} (2 semanas): Elaboración de documentación técnica detallada que describa la arquitectura, uso de la herramienta, decisiones de diseño y ejemplos de generación de EPAs sobre implementaciones concretas. \textbf{Entregable:} Documentación técnica.
\end{enumerate}

El cronograma completo se estima en \textbf{32 semanas}, con una dedicación de entre 12 y 14 horas semanales, lo que representa un esfuerzo total aproximado de entre \textbf{390 y 450 horas}.

\begin{table}[H]
\centering
\begin{tabular}{|c|l|l|c|}
\hline
\textbf{\#} & \textbf{Hito} & \textbf{Entregable} & \textbf{Semanas} \\
\hline
1 & Investigación de Raft & Presentación técnica & 2 \\
2 & Investigación de Mallory & Presentación técnica & 2 \\
3 & \texttt{prepare-binary} & Código funcional & 6 \\
4 & \texttt{run-mallory} & Código funcional & 6 \\
5 & \texttt{generate-epa} (inicial) & Código funcional & 4 \\
6 & \texttt{generate-epa} (mejorado) & Código funcional & 4 \\
7 & Validación y refinamiento de abstracciones & Código funcional & 4 \\
8 & Generación de documentación & Documentación técnica & 2 \\
\hline
\textbf{-} & \textbf{Total} & \textbf{-} & \textbf{32} \\
\hline
\end{tabular}
\caption{Detalle de hitos y duración estimada de Abstraktor}
\label{tab:hitos-abstraktor}
\end{table}

\subsection{Gestión de Calidad y Pruebas}
Se hará un enfoque en la calidad del producto y también del código fuente entregado. Se implementará \textbf{testing automatizado}, ejecutado en \textit{pipelines} de GitHub:
\begin{itemize}
    \item Las pruebas incluirán tests unitarios y de integración para garantizar la estabilidad del sistema.
    \item Se incorporará un linter para asegurar la consistencia y calidad del código.
\end{itemize}

\subsection{Documentación}
Para asegurar la correcta comprensión y mantenimiento del sistema, se generará y actualizará la siguiente documentación:  

\begin{itemize}
    \item \textbf{Documentación técnica}: Detalles de los componentes y arquitectura del sistema.
    \item \textbf{Manual de usuario}: Documentación funcional de como utilizar \textbf{Abstraktor}.
    \item \textbf{Minutas de reuniones}: Registro de decisiones y cambios en la planificación.
\end{itemize}

\subsection{Gestión de Indicadores y Riesgos}
Se realizará una etapa experimental y de investigación extensa para identificar posibles riesgos técnicos y evaluar diferentes soluciones al problema. Se ajustará la planificación según sea necesario:
\begin{itemize}
    \item Se monitoreará el progreso mediante la cantidad de tareas completadas, ponderadas por dificultad, por iteración.
    \item Se mantendrán reuniones regulares dentro del equipo para revisar avances y bloqueos.
\end{itemize}

\subsection{Validación}

Para validar \textbf{Abstraktor}, se realizarán pruebas sobre diversas implementaciones de protocolos distribuidos. Estas implementaciones se seleccionarán en función de su fiabilidad, incluyendo tanto implementaciones correctas como aquellas que contienen errores conocidos.

De esta manera, se podrá verificar la capacidad de \textbf{Abstraktor} para identificar fallas en la lógica de negocio y asegurar que el sistema funcione correctamente.