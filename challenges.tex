\section{Riesgos materializados y lecciones aprendidas}

Esta sección describe los principales riesgos materializados durante el desarrollo de \textbf{Abstraktor}, las estrategias de mitigación aplicadas y las lecciones aprendidas. Los riesgos identificados incluyen incertidumbre técnica en las soluciones de implementación, y desafíos relacionados con la integración y extensión de software legacy.

\subsection{Incertidumbre en la viabilidad de soluciones técnicas}

No existía certeza sobre la viabilidad de las soluciones técnicas necesarias para cumplir con los requisitos funcionales de \textbf{Abstraktor}. La complejidad de la instrumentación de código C/C++, la integración con Mallory y la generación automática de abstracciones presentaban desafíos técnicos sin un enfoque de implementación claro desde el inicio.

Esta incertidumbre requirió experimentación extensiva, resultando en múltiples branches de experimentación y pruebas de concepto. Sin embargo, estas branches no pudieron ser integradas directamente a main debido a la baja calidad del código: contenían valores hardcodeados, lógica ad-hoc, falta de tests y documentación insuficiente.

Para mitigar este riesgo, se adoptó una estrategia de dos etapas: primero se realizó experimentación libre en branches separadas para validar la viabilidad técnica de las soluciones, y luego se reimplementaron los experimentos exitosos aplicando las prácticas de ingeniería de software establecidas antes de integrar al código principal. Este enfoque permitió validar soluciones técnicas sin comprometer la calidad del código base.

\subsection{Falta de tests en software legacy}

El software legacy Mallory, sobre el cual se construyó \textbf{Abstraktor}, no contaba con suficientes tests ni cobertura de código adecuada. Esto dificultó la comprensión del impacto de los cambios al modificar o extender el código. Sin tests apropiados, incluso modificaciones pequeñas representaban el riesgo de introducir bugs o romper funcionalidad existente.

Para mitigar este riesgo, se desarrollaron tests de integración entre \textbf{Abstraktor} y Mallory. Estos tests garantizaron que los cambios no rompieran el flujo de trabajo general y que la funcionalidad clave permaneciera intacta al modificar o extender el código.

\subsection{Falta de documentación en software legacy}

Mallory presentaba documentación muy limitada. Proporcionaba escasa orientación sobre cómo compilar y ejecutar el software, y casi ninguna información sobre sus componentes internos, arquitectura o ubicación de lógica específica en el código.

Como resultado, se invirtió mucho tiempo intentando comprender el sistema, no solo para utilizarlo, sino también para extenderlo con funcionalidad adicional requerida por \textbf{Abstraktor}. 

Esta experiencia evidenció la importancia crítica de la documentación tanto para el desarrollo actual como para la mantenibilidad futura. La falta de documentación incrementó considerablemente el tiempo de onboarding y el esfuerzo necesario para realizar modificaciones en el código legacy.
