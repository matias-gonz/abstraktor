\section{Metodología aplicada}

El desarrollo de \textbf{Abstraktor} se llevó a cabo aplicando prácticas de ingeniería de software modernas, con un enfoque ágil basado en iteraciones cortas y mejora continua. Esta sección describe el contexto en el que se realizó el desarrollo, el proceso aplicado, las herramientas utilizadas y las prácticas de calidad implementadas.

\subsection{Contexto del desarrollo}

Este trabajo se realizó en colaboración con el \textbf{Laboratory on Foundations and Tools for Software Engineering (LaFHIS)} de la Facultad de Ciencias Exactas y Naturales de la Universidad de Buenos Aires. El laboratorio se encuentra investigando activamente la aplicación de \textbf{Enabledness-Preserving Abstractions (EPAs)} como técnica de verificación formal.

Uno de los objetivos del laboratorio es aplicar EPAs a algoritmos distribuidos. \textbf{Abstraktor} surge como respuesta a esta necesidad, proporcionando un framework automatizado para la instrumentación de código, ejecución guiada mediante fuzzing y generación de EPAs a partir de implementaciones reales de protocolos distribuidos.

El equipo de desarrollo estuvo compuesto por dos contribuidores principales: un estudiante de doctorado trabajando en su tesis y un estudiante de ingeniería de software completando su Trabajo Profesional Final. Ambos estudiantes trabajaron guiados por un director de investigación que, en términos del proceso de desarrollo, cumplió el rol de Product Owner. Adicionalmente, el director del Trabajo Profesional Final actuó como facilitador de reuniones y supervisor del desarrollo, asegurando la dirección y calidad del proyecto.

