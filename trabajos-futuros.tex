\section{Trabajos futuros}

El desarrollo de \textbf{Abstraktor} representa un primer paso hacia la automatización de la generación de EPAs para protocolos distribuidos. Esta sección describe las líneas de trabajo futuras identificadas.

\subsection{Experimentación y validación}

El objetivo del laboratorio LaFHIS es utilizar \textbf{Abstraktor} para experimentar con implementaciones reales del protocolo Raft, generar EPAs automáticamente y comparar las abstracciones obtenidas con aquellas derivadas de especificaciones formales del protocolo.

\subsection{Evolución de Abstraktor}

\textbf{Abstraktor} requiere mantenimiento continuo y extensiones para ser utilizable en contextos más amplios. Se planea agregar funcionalidades una vez identificadas nuevas necesidades durante la experimentación.

\subsection{Mejoras técnicas}

Se identificaron áreas técnicas que requieren trabajo adicional:

\begin{itemize}
    \item \textbf{Tests más exhaustivos}: Ampliar la cobertura de tests e incluir los tests de integración en el pipeline de CI/CD.
    
    \item \textbf{Optimización de infraestructura}: Reducir el tiempo de setup de Mallory, que representa un cuello de botella significativo en el ciclo de desarrollo y experimentación.
\end{itemize}


